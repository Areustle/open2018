\documentclass[12pt, letterpaper]{article}
\usepackage[letterpaper, margin=1in]{geometry}
\usepackage[style=numeric-comp]{biblatex}
\usepackage{hyperref}
\hypersetup{
    colorlinks=true,
}
\bibliography{cite}

\title{Best Practices for a Future Open Code Policy for NASA Space Science \\
Response to a Call for White Papers}
\author{Alexander Reustle}
\begin{document}
\begin{center}
\large {Best Practices for a Future Open Code Policy for NASA Space Science \\
Response to a call for White Papers}
\\
\normalsize {Alexander Reustle}

\end{center}

  \begin{abstract}
  Open source code is better code, as it can use the many free tools and services
  now available to the open source community. Access to these tools helps
  teams easily follow best practices for creating, testing, and
  distributing scientific software.
  \end{abstract}

  % \section*{Background}
  % I am a scientific software developer at the Goddard Spaceflight Center working
  % on the SESDA-4 (Space and Earth Science Data Analysis) contract for INNOVIM.
  % My exclusive task is to support the Fermi Science Support Center, maintaining
  % our open data archive and developing science analysis software for the Fermi
  % LAT collaboration. I have six years experience working in this role for Fermi.

  In recent years the popularity of open source software and the infrastructure
  to support it has grown phenomenally. Companies now provide
  services for free to open source code-bases. These services enable
  teams to follow best practices for software development with a minimal
  investment in time, money, and other resources. This brief white paper will
  enumerate a selection of these services and describe how they can benefit
  scientific software development. Each tool is free to use for open
  source projects and teams. While many of the practices described are
  achievable with closed source solutions, none of them are as cheap, fast and
  low-maintenance as their open source alternatives. Some can especially
  benefit code written in academic environments. Even those with NASA funding
  often lack the resources that NASA internal teams have access to. If our goal
  is safer, better, faster, cheaper then open source science code is the way to
  achieve it.

  \section*{\small Github}

  Github\cite{github} is a service that enables developers to save, organize,
  and share open source code for free online. It is a website interface to the
  popular git version control system and enables the following:
  \begin{itemize}
      \item Source code can be easily shared among teams without the need to
        set up and maintain an expensive web server. This enables free and
        easy collaboration.
      \item Published results can reference a tag of the code at a specific
        state, which will never change. Other researchers can replicate past
        findings using that tag,  while the development team makes further
        changes.
      \item Users of the code can contribute to future development by providing
        their own changes, while the owner of the code-base retains ultimate
        control over what is included.
      \item Owners retain full control over official code-bases and can migrate
        to government, academic, or private alternatives to Github freely and
        easily.
  \end{itemize}

  \section*{\small Codecov}

  Testing code is crucial to producing reliable software packages,
  but testing is tedious and often difficult. It can be hard to know whether
  tests cover every part of the code-base, or if some parts remain untested.
  Codecov\cite{codecov} is a service that analyzes open source projects for
  free and reports which parts of the code base is covered by the given tests.

  \section*{\small TravisCI}

  Running tests is its own challenge, especially when the code
  must be tested on many operating systems. By lowering the barriers to
  effective testing, teams can catch more bugs and security vulnerabilities
  before releasing code. TravisCI\cite{travisci} is one of many cloud testing
  tools helping developers write safer, more correct and reliable code.
  \begin{itemize}
      \item Tests are run automatically when a change is seen on Github.
      \item Tests are run in the cloud on multiple operating systems with no
        human interaction needed. Teams can test on some systems they otherwise
        cannot access.
      \item Success and failure reports are generated automatically in an easy
        to read format.
  \end{itemize}

  \section*{\small Conda}

  Installing software of any kind on a user's machine can be a difficult task.
  Different operating systems have different requirements, and all software
  packages depend on other packages to run. For example, graphing code depends
  upon a graphing library to display results. Conda\cite{conda} is a free
  package manager for open source packages. Many other package managers exist
  with differing strengths and weaknesses. Teams can reasonably distribute open
  source software through more than one package manager.
  \begin{itemize}
      \item User installation is as easy as typing \texttt{conda install} into
        a terminal.
      \item Developers tell Conda which other packages their code depends on.
        Conda installs these dependencies before installing the developer's
        package.
      \item Conda supports all programming languages unlike other
        managers which focus on a single language.
      \item Conda supports Linux, Mac, and Windows unlike other
        managers which focus on a single Operating system.
  \end{itemize}

  \
  NASA teams and academic collaborators alike have real opportunities to
  benefit if we shift to a policy that permits and encourages open source
  development. Saving money and reducing overhead will empower teams of all
  sizes and funding levels to write better code more reliably; code that is
  easier to share, use, and improve.
\printbibliography
\end{document}

